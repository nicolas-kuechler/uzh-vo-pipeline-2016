\chapter{Functionality of the VO Pipeline}
\vspace{-10mm} 

\section{Bootstrapping}
The goal of the bootstrapping phase is to initialize a 3D point cloud and to estimate the rotation and translation between two manually selected key frames (frame 1 and 3).

At the beginning \emph{harris features} are detected and matched from both key frames and then the \emph{8-point algorithm} in combination with \emph{RANSAC} is applied. 

In the \emph{8-point algorithm} we estimate the \emph{essential matrix} from eight point correspondences selected uniform at random from the matched features. After recovering rotation and translation, they are used to triangulate a 3D point cloud. In order to validate the pose estimate these 3D points are re-projected and classified as outliers if they exceed a certain re-projection error threshold calculated with \emph{SSD}. 

After 1'200 iterations of \emph{RANSAC}, the pose estimate with the highest number of inliers is kept. All of these inliers are then used in the \emph{8-point algorithm} to refine the pose by minimizing the overdetermined solution. The final initial 3D point cloud is then triangulated with the refined pose.

\section{Continuous Operation}
As subsequent frames are processed in a Markovian fashion, meaning that information from the previous frame is sufficient to compute all necessary variables of the subsequent frame. \par
When processing a new frame, first key points (matched with landmarks) are tracked from the previous image to the next image using a KLT tracker [ref] implemented by MATLAB. Due to the large displacement of key points across images 3 pyramidal levels and a patch size of 31 x 31 pixels are used. \par
In the second step the new camera homography is computed using the new 2D-3D correspondence from the tracked key points. This can be efficiently computed using a MATLAB function. This function uses the P3P algorithm in combination with the M-estimator sample consensus (MSAC) to remove outliers. Using only three points allows few iterations to achieve high confidence (??\%). Non-tracked and outlier key points are removed with their corresponding landmarks.\par
Next new landmarks are triangulated from candidate key points which were tracked over several steps. Triangulation is performed between the track end and start. This means every track start and the corresponding homographies are saved. New landmarks are triangulated asynchronously and only if the angle between the bearing vectors at the track start and end exceeds ??. Candidate tracks are removed if they cannot be tracked and landmarks are removed if their reprojection error exceeds ?? pixels or are triangulated behind the camera.\par
This candidate loss through tracking and triangulation means new candidates must be added continuously. ?? new candidates are added at each frame using a Harris corner detector and capped at ??. To achieve a more robust behaviour certain areas of the image are suppressed before Harris corners are extracted. More information can be found in chapter 3. 

\section{Plots}
For visualization indicative parameters of the performance are plotted. A standard plot is seen in Fig. ??.  

% Include a figure of a standard plot
\begin{figure}
  \includegraphics[width=\linewidth]{}
  \caption{A boat.}
  \label{fig:boat1}
\end{figure}

In the top left the current image is displayed with marked key points. The large green crosses represent key points which have matched landmarks associated to them. The smaller blue crosses are candidates which have been tracked over several frames. \par
On the bottom left the number of candidates and matched key points can be seen over the last 20 frames and gives an indication of the accuracy of the estimation since very low levels of matches lead to high errors. \par 
The bottom middle plot then shows the overall trajectory of the camera. The right plot shows a local magnification of the trajectory an displays the landmarks as they are perceived in the environment.