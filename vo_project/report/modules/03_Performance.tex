\chapter{Performance of our VO Pipline}
\label{performance}

\section{Effect of Bundle Adjustment}
- ref to fig of aligned trajectory w/ and w/o BA
- timing 
- discussion
- Bundle Adjustment decreases the maximum framerate by a factor x
- #toAdjustOrNotToAdjust

\section{Error Analysis of Params}

In order to do a thorough optimization, a total of 15 parameters would have to varied simultaneously. 
This would lead to an intractable number of valid combinations. To curb the computational burden we proceeded to group the parameters into three main clusters:

\begin{itemize}
    \item harris corner detection (harris\_kappa, harris\_patch\_size)
    \item candidate adding (candidate\_cap, candidate\_add, nonmaximum\_supression\_radius)
    \item repeated optimization with most sensitive parameters.
\end{itemize}

First, a sane parameter configuration was guessed for each cluster. Then, groups were fine tuned in a parallel simulation setup \coderef{simTaskScheduler.m]}. Parameters were selected whenever a significant impact on accuracy (deviation from ground truth, see chapter \ref{simulation}) and robustness (successful completion of simulation challenge with varying other parameters) was detected.

\subsection{Harris Tuning}
From a qualitative observation of the trajectories, we could observe that features detected with a patch size of 9 and a kappa of 0.08 result in low tracking loss when applied on the provided data sets.

\subsection{Candidate Reinforcement Tuning (adding of candidates)} 
Error analysis over 4 params: nonmaximum\_supression\_radius,candidate\_cap,surpress\_existing\_matches,add\_candidate_each\_frame.

Results:
\begin{itemize}
    \item Suppress\_existing\_matches strongly improves the algorithm stability.
    \item The candidate\_cap has negligible impact on the precision when above 500.
    \item Large variation were found in nonmaximum\_supression_radius (selected for parameter tuning).
    \item Large variations were found in add\_candidate\_each\_frame (selected for parameter tuning).
\end{itemize}

\subsection{Significant Parameter Tuning}  
From empirical studies (trial and error) sensitive (high impact) parameters were identified and interesting ranges were selected for the final optimization.

The following parameters were considered especially relevant (as seen in the reinforcement tuning):

\begin{itemize}
    \item nonmaximum\_supression\_radius
    \item add\_candidate\_each\_frame
    \item triangulate\_max\_repr\_error
    \item critical\_kp
    \item tracker\_max\_bidirectional\_error
\end{itemize}

% Discussion:

% Measure which are the best combinations --> do simulation for this cluster
% Results - best combinations: (make table)
%12, 150, 5, 100, 5 
%10, 200, 5, 0, 2,1
%10, 200, 5; 0, 5

% How optimal are these results

average orientation error: minimum: 1.97 maximum: 23.79, median: 2.89
average location error: minimum: 6664.9 maximum: 23187.5 , median: 16059.0

% Results - quotient between maximum and minimum  > 3 --> strong minimum

% How robust are these results
% Results - Histograms of failures and 100 (or 200 or 10) best --> robust and doesn't fail easily

% --> in the end parallel discussion of rotation error.

% Conclusion: optimization of parameters for rotation yield parameters which are not dominantly better than all others. Therefore rely on optimal parameters for location error.

Scale drift is a problem that cannot be solved with parameter tuning and must be addressed with BA.



