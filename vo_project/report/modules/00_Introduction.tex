\chapter{Introduction}
\vspace{-10mm}
\section{Monocular Visual Odometry}
The following report describes a completely monocular visual odometry (VO) pipeline. It relies on the input of a single camera feed to estimate a trajectory which is valid up to an unknown scale factor.

\section{Parameter Settings}
\label{params}
Before starting our VO pipeline by executing the file \emph{main.m}, the tester has to choose the data set, whether or not bundle adjusted and/or trajectory alignment to the ground truth should be performed. The last two are controlled with variables \emph{align\_to\_ground\_truth} and \emph{bundle\_adjustment} while the first can be used to distinguish between the KITTI (ds=0), Malaga (ds=1), Parking (ds=2) and our own (ds=3) data set. (\ref{dataset})

\section{Additional Work}
The successful implementation of the VO pipeline involved the tuning of several parameters which were vital to the stable performance. This was done by performing a quantitative simulation in order to find the combination that leads to the best trajectory estimate. For this a performance metric consisting of the average trajectory deviation was defined and used to simulate several combinations of the parameters. More on this analysis can be found in section \ref{simulation}. \par
For the purpose of testing these ideal parameters we recorded an own video and integrated it into the existing VO pipeline. This is an important test, because it assured us that we were not over-fitting our parameters to the benchmark data sets. More information on the data set can be found in section \ref{dataset}. \par
Unfortunately scale drift is inevitable in a VO pipeline due to errors in the pose and landmark estimates. To combat this drift multiple views can be refined using bundle adjustment (BA). We integrated a form of sliding window and overlapping bundle adjustment into the existing pipeline in order to refine both structure and motion. The adjusted estimates are significantly more accurate but at the cost of computational performance. For further information consult section \ref{bundle adjustment}.
